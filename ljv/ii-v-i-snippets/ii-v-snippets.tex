% Created 2025-01-09 Thu 18:25
% Intended LaTeX compiler: lualatex
\documentclass[11pt]{article}
\usepackage{fontspec}
\usepackage{graphicx}
\usepackage{lilyglyphs}
\usepackage{graphicx}
\usepackage{longtable}
\usepackage{wrapfig}
\usepackage{rotating}
\usepackage[normalem]{ulem}
\usepackage{amsmath}
\usepackage{amssymb}
\usepackage{capt-of}
\usepackage{hyperref}
\usepackage[cm]{fullpage}
\usepackage[headheight=15pt, headsep=10pt, top=1in, bottom=1in, left=0.75in, right=0.75in]{geometry} % Ensure sufficient header space
\usepackage{fancyhdr}
\pagestyle{fancy}
\fancyhf{}
\fancyhead[L]{\textbf{ii-V-I Snippets}} % Left header with title
\fancyhead[R]{\textbf{Bartev - Lesson 26 (2024-12)}} % Right header with author
\fancyfoot[C]{\thepage}
\fancyfoot[R]{Printed \today} % Right footer with today's date
\renewcommand{\headrulewidth}{0.4pt} % Optional: Add a horizontal rule below the header
\makeatletter
\let\ps@plain\ps@fancy % Apply "fancy" style to the first page
\let\maketitle\relax % Suppress default title/author rendering
\makeatother
\author{Bartev}
\date{\today}
\title{ii-V Snippets}
\hypersetup{
 pdfauthor={Bartev},
 pdftitle={ii-V Snippets},
 pdfkeywords={},
 pdfsubject={},
 pdfcreator={Emacs 29.4 (Org mode 9.6.15)}, 
 pdflang={English}}
\begin{document}

\maketitle

\section*{Variations on chord changes}
\label{sec:orgc899500}

\subsection*{Major ii-V-I}
\label{sec:org3ac76b5}
\begin{center}
\includegraphics[width=.9\linewidth]{major_ii_v_i.pdf}
\end{center}

\subsection*{Minor ii-V-I}
\label{sec:orgf58e056}

Shortcut, think harmonic minor all the way through. (\flat 3 \& \flat 6)

\begin{center}
\includegraphics[width=.9\linewidth]{minor_ii_v_i.pdf}
\end{center}

For the half-dim chord, try the Locrian or Locrian nat 2 scales.

\begin{center}
\includegraphics[width=.9\linewidth]{locrian.pdf}
\end{center}

For the G7\flat 9, try the Phrygian Dominant or altered scales.

Usually on a Dom7\flat9 there is a \flat13 (\flat6).

We can think of this as the C harmonic minor (\flat 3 \flat 6) starting on the 5th (G).

The G altered scale starts on G, then every other note is lowered a 1/2 step.

\begin{center}
\includegraphics[width=.9\linewidth]{phryg_dom.pdf}
\end{center}

For the Cm7, use the Dorian or melodic minor scales

\begin{center}
\includegraphics[width=.9\linewidth]{dorian_melodic_minors.pdf}
\end{center}

\subsection*{iii-VI-ii-V-I}
\label{sec:org5142712}
Can be found in
\begin{itemize}
\item Green Dolphin Street
\item There Will Never Be Another You
\end{itemize}

\begin{center}
\includegraphics[width=.9\linewidth]{iii-vi-ii-v.pdf}
\end{center}

\subsection*{Tritone sub}
\label{sec:org34d3330}
Replace the V7 with the V7 a tritone away.

We can also transpose the ii-7 by a tritone.

\begin{itemize}
\item A tritone is a dim 5th (3 whole steps).

\item It is 1/2 way to the octave.
\end{itemize}

\begin{center}
\includegraphics[width=.9\linewidth]{tritone_sub.pdf}
\end{center}

Common notes in substituted chords

\begin{center}
\includegraphics[width=.9\linewidth]{tritone-common-notes.pdf}
\end{center}

\subsection*{Backdoor Dominant}
\label{sec:orgbc64953}

Can also think of this as a minor 3rd sub.

Replace the ii and V chords with chords a m3 higher

\begin{center}
\includegraphics[width=.9\linewidth]{backdoor.pdf}
\end{center}

Notice the common notes in the arpeggios

\begin{center}
\includegraphics[width=.9\linewidth]{backdoor-common-notes.pdf}
\end{center}

\section*{Basic ii-V-I phrases}
\label{sec:org51638b3}
\subsection*{In E-flat}
\label{sec:org3a0248b}

\begin{center}
\includegraphics[width=.9\linewidth]{e-flat.pdf}
\end{center}

\subsection*{G maj}
\label{sec:orgc1debc5}
\begin{center}
\includegraphics[width=.9\linewidth]{g_maj.pdf}
\end{center}

\subsection*{G maj (variation 2)}
\label{sec:orgeeba5e0}
\begin{center}
\includegraphics[width=.9\linewidth]{g_maj_v2.pdf}
\end{center}

\section*{Tritone sub phrases (Dom chord only)}
\label{sec:org27bf2bf}

\subsection*{Tritone in Cmaj}
\label{sec:orgd8a2ad9}
\begin{center}
\includegraphics[width=.9\linewidth]{tritone-c-maj.pdf}
\end{center}
\end{document}
